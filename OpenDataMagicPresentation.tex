\environment envpresentation2


\setlayerframed[footy][frame=off, x=.05\pagewidth, y=.0075\pageheight, align=flushleft, width=.8\pagewidth, height=.147\pageheight]%
{\color[white]{\setupbodyfont[50pt]\setupinterlinespace[line=0pt]\setupwhitespace[none]Open Data, The Reproducibility Problem, and Sympathetic Magic
\vfill}}


\setlayerframed[footy][frame=off, x=.05\pagewidth, y=.1245\pageheight, align=center]%
{\color[white]{\setupbodyfont[20pt]\setupinterlinespace[line=0pt]\setupwhitespace[none]Brian Ballsun-Stanton (Department of Ancient History)}}


\definebar[a][color=blue]
\starttext

\vfill

By: Dr Brian Ballsun-Stanton 

(Twitter: @DenubisX)

Technical Director \crlf Field Acquired Information Management Systems Project

Department of Ancient History \crlf Macquarie University

\vfill

{\tfxx This presentation taken from notes on a forthcoming chapter called: {\it Open Data and the Danger of Sympathetic Magic} by: Dr Brian Ballsun-Stanton, Macquarie University, Technical Director of the FAIMS project and Georgia Burnett, Macquarie University. Typeset with \ConTeXt. \crlf Presentation source available on github.com/FAIMS/OpenDataPresentation.
}
\section{Collecting data is hard, reusing it is harder}

\startitemize
\item I help people deal with data collection while offline
\item Collecting data in the middle of nowhere is a huge pain
\item Even bigger challenge: working with already collected data
\stopitemize

\section{Show of hands:}

How many of you have tried to use someone else's data?


\section{Current Challenges of Open Data}

\startitemize
\item No rewards for quality open data releases
\item What's a "knowledge-creation framework?"
\item Write-once Read-once
\item No Context
\stopitemize

\section{Standing on the Shoulders of Giants}
... should include standing on their data.

\startitemize
\item Valuable data is valuable to all
\item If only arguments of value, so much is lost
\stopitemize

\section{Sympathetic Magic}

\startblockquote
"PERHAPS the most familiar application of the {\tfa\em\it principle that like produces like} is the attempt which has been made by many peoples in many ages to injure or destroy an enemy by injuring or destroying an image of him, in the belief that, just {\tfa\em\it as the image suffers, so does the man}, and that when it perishes he must die. "
\stopblockquote

Sir J.G. Frazier, The Golden Bough (1922)
\section{What?}

\startitemize
\item Data is the product of methodology and analysis
\item Methodology of that which to be recorded
\item Methodology of how to record observations
\item Methodology of analysis
\item Methodology of the mechanics of analysis
\stopitemize

\section{Mere presentation of data is sympathetic magic: }

The manipulation of data without 
tests, context and mechanisms is 
the manipulation of the data's 
sympathetic image.

\section{Data as a problem of sympathetic magic}

\startitemize
\item We supply the image of the data, expecting it to be useful
\item The true utility is the structure of the data and its methods of generation, storage, and analysis, not just a magic charm CSV file
\stopitemize

\section{Goals of open data}

\startitemize
\item Complete datasets online in curated spaces
\startitemize
\item Includes metadata for more accurate reuse
\item Curation prevents data format and structure from becoming obsolete or deteriorating by publishing data alongside scholarship
\stopitemize
\item Data that is free of cost
\item Data that is free of copyright/contractual obligation
\stopitemize

\section{Tests for quality}

\startitemize
\item Test: Is a re-user able to add to a dataset without outside consultation?
\item Test: can the re-user re-run initial data analysis of authors?
\stopitemize


\section{Goals of open, useful, data}


\startitemize
\item What does the data mean to those who recorded it?
\startitemize
\item Documentation of mechanisms of recording and analysis
\item Documentation of judgement calls and design decisions
\stopitemize
\item Addresses epistemological questions
\item Does not require out-of-band knowledge
\item Does not require knowledge of conclusions before run
\stopitemize

\section{Ethical imperatives}


\startitemize
\item Publishing data paid for by taxpayers for the "public good"
\item Reasons not to hoard data
\item Long-term, sustainable data preservation 
\stopitemize

\section{Bigger Picture}

\startitemize
\item Change in publishing climate
\item This publishing-of-analytic-programs work is adjunct to a paper
\item We do this because we want other people to steal our research methods for the good of knowledge
\item Ideas are cheap, execution is hard
\item Research is very expensive
\stopitemize



\section{Conclusion}


\startitemize
\item Reusing data to answer novel questions without asking the original researchers for help (or a thumb drive) is very hard.
\item Publishing only a superficial image of data, however, is sympathetic magic.
\stopitemize


\vfill 

{\tfxx Presentation source available on github.com/FAIMS/OpenDataPresentation.}






\stoptext

